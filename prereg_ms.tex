\documentclass[]{article}
    \usepackage{lmodern}
    \usepackage{amssymb,amsmath}
\usepackage{ifxetex,ifluatex}
\usepackage{fixltx2e} % provides \textsubscript
\ifnum 0\ifxetex 1\fi\ifluatex 1\fi=0 % if pdftex
\usepackage[T1]{fontenc}
\usepackage[utf8]{inputenc}
  \else % if luatex or xelatex
\ifxetex
\usepackage{mathspec}
\usepackage{xltxtra,xunicode}
\else
  \usepackage{fontspec}
\fi
\defaultfontfeatures{Mapping=tex-text,Scale=MatchLowercase}
\newcommand{\euro}{€}
        \fi
% use upquote if available, for straight quotes in verbatim environments
\IfFileExists{upquote.sty}{\usepackage{upquote}}{}
% use microtype if available
\IfFileExists{microtype.sty}{%
  \usepackage{microtype}
  \UseMicrotypeSet[protrusion]{basicmath} % disable protrusion for tt fonts
}{}
  \usepackage[left=2.5in,bottom=1.25in,top=1.25in,right=1in]{geometry}
  \ifxetex
\usepackage[setpagesize=false, % page size defined by xetex
            unicode=false, % unicode breaks when used with xetex
            xetex]{hyperref}
\else
  \usepackage[unicode=true]{hyperref}
\fi
\hypersetup{breaklinks=true,
bookmarks=true,
pdfauthor={},
pdftitle={Preregistration: Bat Community Structure In the Cypress Hills pre/post White-Nose Syndrome},
colorlinks=true,
citecolor=blue,
urlcolor=blue,
linkcolor=magenta,
pdfborder={0 0 0}}
\urlstyle{same}  % don't use monospace font for urls
\setlength{\parindent}{0pt}
\setlength{\parskip}{6pt plus 2pt minus 1pt}
\setlength{\emergencystretch}{3em}  % prevent overfull lines
\providecommand{\tightlist}{%
\setlength{\itemsep}{0pt}\setlength{\parskip}{0pt}}
\setcounter{secnumdepth}{0}

% Customization for cos_prereg
\usepackage{longtable,booktabs,threeparttable,tabularx}
\linespread{1.5}
\newcounter{question}
\setcounter{question}{0}

%%% Use protect on footnotes to avoid problems with footnotes in titles
\let\rmarkdownfootnote\footnote%
\def\footnote{\protect\rmarkdownfootnote}

%%% Change title format to be more compact
\usepackage{titling}

\def\changemargin#1#2{\list{}{\rightmargin#2\leftmargin#1}\item[]}
\let\endchangemargin=\endlist

% Create subtitle command for use in maketitle
\newcommand{\subtitle}[1]{
\posttitle{
\begin{center}\large#1\end{center}
}
}

\setlength{\droptitle}{-2em}
\title{Preregistration: Bat Community Structure In the Cypress Hills
pre/post White-Nose Syndrome}
\pretitle{\begin{changemargin}{-8pc}{0pc} \centering\large Preregistration\\ \Huge}
\posttitle{\end{changemargin}}
  \author{
          First Author\textsuperscript{1},
          Javad Meghrazi\textsuperscript{2},
          Hannah Wilson\textsuperscript{3},
          Sophia Fan\textsuperscript{4}          \\ \vspace{0.5cm}
              \textsuperscript{1} Wilhelm-Wundt-University\\
              \textsuperscript{2} Konstanz Business School\\
              \textsuperscript{3} University of Regina\\
              \textsuperscript{4} University of British Columbia      }

  \def\affdep{{"", "", "", ""}}%
  \def\affcity{{"", "", "", ""}}%
  \preauthor{\begin{changemargin}{-8pc}{0pc} \centering\large}
  \postauthor{\end{changemargin}}
\date{04. October 2021}
\predate{\begin{changemargin}{-8pc}{0pc} \centering\large\emph}
\postdate{\end{changemargin}}
\usepackage{fancyhdr}
\pagestyle{fancy}
\renewcommand{\headrulewidth}{0pt}
\lhead{}
\rhead{\large\textsc{\MakeLowercase{PreReg: Bat Community Structure in
Cypress}}}



% Title settings
\usepackage{titlesec}
\titleformat{\section}[display]{\bfseries\Large}{\thesection}{}{}[]
\titlespacing{\section}{0pc}{*3}{*1.5}
\titleformat{\subsection}[leftmargin]{\titlerule\bfseries\filleft}{\thesubsection}{.5em}{}
\titlespacing{\subsection}{8pc}{5ex plus .1ex minus .2ex}{1.5pc}
  

% Redefines (sub)paragraphs to behave more like sections
\ifx\paragraph\undefined\else
\let\oldparagraph\paragraph
\renewcommand{\paragraph}[1]{\oldparagraph{#1}\mbox{}}
\fi
\ifx\subparagraph\undefined\else
\let\oldsubparagraph\subparagraph
\renewcommand{\subparagraph}[1]{\oldsubparagraph{#1}\mbox{}}
\fi


\begin{document}
\maketitle
\vspace{2pc}


\newcommand\Question[2]{%
   \leavevmode\par
   \stepcounter{question}
   \noindent
   \textbf{\thequestion. #1}. #2\par}

\newcommand\Answer[1]{%
    \noindent
    \textit{Registered response}: #1\par}
    
\hypertarget{study-information}{%
\section{Study Information}\label{study-information}}

\hypertarget{title}{%
\subsection{Title}\label{title}}

Preregistration: Bat Community Structure In the Cypress Hills pre/post
White-Nose Syndrome The effects of white-nose syndrome (WNS) on bat
community structure in Cypress Hills, Saskatchewan.

\hypertarget{description}{%
\subsection{Description}\label{description}}

Enter your response here.

White-nose syndrome, first detected in New York during the winter of
2006, is an infectious fungal disease that has since affected and killed
more than 6 million hibernating bats in eastern North America Jachowski
et al. (2014). Notably, the disease has had a pronounced effect on the
little brown bat, \emph{Myotis lucifugus}, where it's predicted that
regional population collapse and extirpation could occur as soon as
within the next 16 years Frick et al. (2010). Because WNS affects bat
species differentially, community-level monitoring studies are needed to
fully appreciate and mitigate consequences for community structure and
ecosystem function Hoyt, Kilpatrick, \& Langwig (2021). Recently,
researchers have identified instances of WNS in Cypress Hills,
Saskatchewan, indicating concerning disease spread into central Canada
(CITATION?). This study aims to inform conservation efforts by compiling
and comparing bat community data from before and after the introduction
of WNS in Cypress Hills, in addition to constructing population
projection models for each species.

\hypertarget{hypotheses}{%
\subsection{Hypotheses}\label{hypotheses}}

Enter your response here.

\textbf{Directional}: Like other parts of eastern North America, we
expect to see pronounced population \textbf{declines} in little brown
bats (\emph{Myotis lucifugus}) along with declines in most other bat
species within an infected community.

\textbf{Directional}: We expect to see an \textbf{increase} in
silver-haired bats (\emph{Lasionycteris noctivagans}) (CITATION?).

\hypertarget{design-plan}{%
\section{Design Plan}\label{design-plan}}

\hypertarget{study-type}{%
\subsection{Study type}\label{study-type}}

\textbf{Observational Study}. Data is collected from study subjects that
are not randomly assigned to a treatment. This includes surveys, natural
experiments, and regression discontinuity designs.

\hypertarget{blinding}{%
\subsection{Blinding}\label{blinding}}

No blinding is involved in this study.

\hypertarget{study-design}{%
\subsection{Study design}\label{study-design}}

Timing: The Bat Community in the Cypress Hills has been monitored since
1991, and we will continue to monitor the community until 2030.
Monitoring will start around mid-June every year, depending upon permit
permissions. Monitoring will finish in mid-August.

Study sites: Researchers will place mist nets at various points along
Battle Creek in the Cypress Hills West Block. We will choose these sites
based on records from previous capture data. We will continue to return
to these sites each year until 2030. We will also attempt to net at each
site at similar dates each year.

Field methods: We will raise 1-3 mist nets across Battle Creek at each
net site. Number of nets will be chosen based on personnel, time, and
previous net set-ups. Each mist net will be raised at last light and we
will monitor net activity for three hours per night. Every hour we will
record ambient temperature, cloud cover and wind speed. We will close
nets if it is raining or wind speed is above 20 km/s.

We will check the nets every 10 minutes for one hour after last light,
and every 15 minutes for the two proceeding hours. If a bat is captured
it will be extracted and moved away from the mist nets in order to
reduce other bats being attracted to the distress call of the captured
bat. We will take data on each captured bats species, age, sex, and
mass. Bats will be released at most half an hour after capture.

\hypertarget{randomization}{%
\subsection{Randomization}\label{randomization}}

Not applicable

\hypertarget{sampling-plan}{%
\section{Sampling Plan}\label{sampling-plan}}

\hypertarget{existing-data}{%
\subsection{Existing data}\label{existing-data}}

\textbf{Registration prior to analysis of the data}. As of the date of
submission, the data exist and you have accessed it, though no analysis
has been conducted related to the research plan (including calculation
of summary statistics). A common situation for this scenario when a
large dataset exists that is used for many different studies over time,
or when a data set is randomly split into a sample for exploratory
analyses, and the other section of data is reserved for later
confirmatory data analysis.

\hypertarget{explanation-of-existing-data}{%
\subsection{Explanation of existing
data}\label{explanation-of-existing-data}}

Capture data from bats in the Cypress Hills has been collected since
1991, and so we have data for this project from 1991 until now. The
existing data was collected by other researchers and was used for many
different studies over time. These researchers left their raw data
available for others to use. Currently, only one of the project members
has seen the original data. However, they have only seen the data from
1991-1992, and the 2019 dataset and they have not conducted any prior
analyses or investigations of it. The other project members have not
seen any of the existing data yet. Furthermore, this project assesses
community structure changes before and after the detection of White-nose
syndrome in Saskatchewan until 2030. We will not analyze the data until
after 2030.

\hypertarget{data-collection-procedures}{%
\subsection{Data collection
procedures}\label{data-collection-procedures}}

Monitoring will start around mid-June every year, depending upon permit
permissions. Monitoring will finish in mid-August. We will monitor
community structure using the capture data we collect from mist nets.

We will set up 1-3 mist nets at one site each night. Nets will be set up
across Battle Creek. Nets will be at least 1m tall, and wide enough the
cover the width of the creek. Nets will be set up at transition sites
where the flyway above Battle Creek transitions from a cluttered
environment with lots of debris to open sky. Nets will also be placed
over calm sections of water, without ripples.

Each mist net will be raised at last light and we will monitor net
activity for three hours per night. Every hour we will record ambient
temperature, cloud cover and wind speed. We will close nets if it is
raining or wind speed is above 20 km/s.

We will check the nets every 10 minutes for one hour after last light,
and every 15 minutes for the two proceeding hours. If a bat is captured
it will be extracted and moved away from the mist nets in order to
reduce other bats being attracted to the distress call of the captured
bat. We will take data on each captured bats species, age, sex, and
mass. Bats will be released at most half an hour after capture.

We will continue to return to these sites each year until 2030. We will
also attempt to net at each site at similar dates each year.

\hypertarget{sample-size}{%
\subsection{Sample size}\label{sample-size}}

Our target sample size is 4000 individuals. However, we do not have a
pre-determined count and cannot control the number of bats we capture
each year.

\hypertarget{sample-size-rationale}{%
\subsection{Sample size rationale}\label{sample-size-rationale}}

We do not have control over how many bats we will capture a night. We
are assuming based anecdotal knowledge and previous experience that in
three months we will capture 100-200 bats, and after forty years of data
collection we will collect around 4000 bats.

\hypertarget{stopping-rule}{%
\subsection{Stopping rule}\label{stopping-rule}}

Data collection will end each year in mid-August. The field station
where researchers live during the field season is owned by the
University of Regina, who use the station for an undergraduate class
during the last week of August. The field station is closed after that,
so data collection will stop mid-August

The project overall will end in 2030. White-Nose Syndrome was first
detected in Saskatchewan in 2021. Ending the project 10 years after WNS
was first detected allows us to observe bat community structure at the
start of the disease, directly after it should have killed the largest
number of bats, and ten years later when it should have become endemic
to the population.

\hypertarget{variables}{%
\section{Variables}\label{variables}}

\hypertarget{manipulated-variables}{%
\subsection{Manipulated variables}\label{manipulated-variables}}

Not applicable

\hypertarget{measured-variables}{%
\subsection{Measured variables}\label{measured-variables}}

-Bat species -year captured -Date captured (yyyy-mm-dd) -capture site
(latitude and longitude) -time caught(?) -number of individuals captured

\hypertarget{indices}{%
\subsection{Indices}\label{indices}}

Enter your response here.

\hypertarget{analysis-plan}{%
\section{Analysis Plan}\label{analysis-plan}}

\hypertarget{statistical-models}{%
\subsection{Statistical models}\label{statistical-models}}

Enter your response here.

\hypertarget{transformations}{%
\subsection{Transformations}\label{transformations}}

Enter your response here.

\hypertarget{inference-criteria}{%
\subsection{Inference criteria}\label{inference-criteria}}

\hypertarget{data-exclusion}{%
\subsection{Data exclusion}\label{data-exclusion}}

Enter your response here.

\hypertarget{missing-data}{%
\subsection{Missing data}\label{missing-data}}

Enter your response here.

\hypertarget{exploratory-analyses-optional}{%
\subsection{Exploratory analyses
(optional)}\label{exploratory-analyses-optional}}

Enter your response here.

\hypertarget{other}{%
\section{Other}\label{other}}

\hypertarget{other-optional}{%
\subsection{Other (Optional)}\label{other-optional}}

Enter your response here.

\hypertarget{references}{%
\section{References}\label{references}}

\hypertarget{section}{%
\subsection{}\label{section}}

\vspace{-2pc}
\setlength{\parindent}{-0.5in}
\setlength{\leftskip}{-1in}
\setlength{\parskip}{8pt}

\noindent

\hypertarget{refs}{}
\begin{CSLReferences}{1}{0}
\leavevmode\hypertarget{ref-frick2010a}{}%
Frick, W. F., Pollock, J. F., Hicks, A. C., Langwig, K. E., Reynolds, D.
S., Turner, G. G., \ldots{} Kunz, T. H. (2010). An Emerging Disease
Causes Regional Population Collapse of a Common North American Bat
Species. \emph{Science}, \emph{329}(5992), 679--682.
doi:\href{https://doi.org/10.1126/science.1188594}{10.1126/science.1188594}

\leavevmode\hypertarget{ref-hoyt2021}{}%
Hoyt, J. R., Kilpatrick, A. M., \& Langwig, K. E. (2021). Ecology and
impacts of white-nose syndrome on bats. \emph{Nature Reviews
Microbiology}, \emph{19}(3), 196--210.
doi:\href{https://doi.org/10.1038/s41579-020-00493-5}{10.1038/s41579-020-00493-5}

\leavevmode\hypertarget{ref-jachowski2014}{}%
Jachowski, D. S., Dobony, C. A., Coleman, L. S., Ford, W. M., Britzke,
E. R., \& Rodrigue, J. L. (2014). Disease and community structure:
white-nose syndrome alters spatial and temporal niche partitioning in
sympatric bat species. \emph{Diversity and Distributions}, \emph{20}(9),
1002--1015.
doi:\href{https://doi.org/10.1111/ddi.12192}{10.1111/ddi.12192}

\end{CSLReferences}

\end{document}